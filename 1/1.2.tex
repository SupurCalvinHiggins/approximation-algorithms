\documentclass{article}

\usepackage{amsmath}
\usepackage{amssymb}
\usepackage{amsthm}

\newtheorem*{theorem*}{Theorem}
\newtheorem*{lemma*}{Lemma}

\newcommand{\opt}{\mathrm{OPT}}

\setlength{\parindent}{0pt}

\begin{document}

\section*{Exercise 1.2}

\subsection*{Problem}

Design a factor 2 approximation for the problem of finding a minimum cardinality 
maximal matching in an undirected graph $G = (V, E)$. \newline

\textbf{Matching.} A matching $M \subseteq E$ in a graph $G = (V, E)$ is a set of edges
such that each $v \in V$ is incident to at most one edge $e \in M$. \newline

\textbf{Maximal Matching.} A maximal matching is a set of edges $M \subseteq E$ such 
that \begin{enumerate}
    \item $M$ is a matching.
    \item Every $M' \supset M$ is not a matching. 
\end{enumerate}

\textbf{Maximum Matching.} A maximum matching $M^* \subseteq E$ is the largest possible
maximal matching.

\subsection*{Solution}

Start with an empty matching $M$. While $M$ is not maximal, add some edge to $M$ such 
that $M$ is still a matching. Output the maximal matching $M$.

\begin{lemma*}
    Every maximal matching has at least $\frac{|M^*|}{2}$ edges.
\end{lemma*}

\begin{proof}
    We claim that one vertex from each edge in $M^*$ must be matched in any maximal 
    matching $M$. For sake of contradiction, suppose not. Then there exists some edge
    $(u, v) \in M^*$ such that no edges in $M$ are incident to $u$ or $v$. Then 
    $M \cup \{(u, v)\}$ is a matching so $M$ is not maximal. However, $M$ is maximal so 
    this is a contradiction. Thus one vertex from each edge in $M^*$ must be in $M$ so 
    there must be at least $\frac{|M*|}{2}$ edges in $M$.
\end{proof}

\begin{theorem*}
    Any maximal matching is a factor 2 approximation.
\end{theorem*}

\begin{proof}
    By the lemma, $\frac{|M^*|}{2} \leq |\opt|$ so $|M^*| \leq 2 |\opt|$. Since $M^*$ is
    the largest possible maximal matching, $|M| \leq |M^*|$. Hence
    \begin{align*}
        |M| \leq |M^*| \leq 2 |\opt|
    \end{align*}
    as desired.
\end{proof}

\subsection*{Insights}

Sometimes, all of the reasonable solutions are good approximations.

\end{document}
