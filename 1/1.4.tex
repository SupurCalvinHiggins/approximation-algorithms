\documentclass{article}

\usepackage{amsmath}
\usepackage{amssymb}
\usepackage{amsthm}

\newtheorem*{theorem*}{Theorem}
\newtheorem*{lemma*}{Lemma}

\newcommand{\opt}{\mathrm{OPT}}

\setlength{\parindent}{0pt}

\begin{document}

\section*{Exercise 1.4}

\subsection*{Problem}

Perhaps the first strategy ones tries when designing an algorithm for an optimization 
problem is the greedy strategy. For the cardinality vertex cover problem, this would 
involve iteratively picking a maximum degree vertex and removing it, together with 
edges incident at it, until there are no edges left. Show that this approximation 
algorithm achieves an approximation guarantee of $O(\log n)$. Give a tight example for 
this algorithm.

\subsection*{Solution}

% define the cost of covering an edge to be the 1/deg(v) where v was the vertex
% that first covered the edge
% consider iteration k + 1 of the algorithm


% cost(e_k) <= OPT_k / remaining_k <= OPT / remaining_k <= OPT / n - k + 1

% remaining_k <= n - k + 1

% actually by def of e_k, remaining k = n - k + 1

% minimum cost to cover the remaining edges <= OPT
% then the average cost to cover remaining edges is <= OPT / (n - k) 
% since at least k edges have been covered so there are at most n - k remaining
% at least k edges have been covered

% summing over cost(e_k) gives an upper bound of H_n OPT
% and H_n is O(log n)

% tight example...
% probably need to force the choice of vertex at each step
% maybe a star graph
% in center, vertex of degree n
% at each end, vertex of degree n - 1
% and so on
% 1 of n
% n of n - 1
% n * (n - 2) of n - 2
% n * (n - 2) * (n - 4) of n - 3
% ...

% in center vertex of degree n + 1
% at each end, vertex of degree n - 1
% at each end, vertex of degree n - 3
% n + 1 of n - 1

% tree
% n then n-1 then n-2 and so on of children
% how many nodes
% 1 + n + n(n-1) + n(n-1)(n-2) + ... + n(n-1)...2 + n!
% algorithm takes everything but the last layer 
% 1 + n + n(n-1) + n(n-1)(n-2) + ... + n(n-1)...2
% optimal is every other
% n + n(n-1)(n-2) + ...

% the algorithm takes all of them
% optimal doesnt take outermost, then third outermost, etc

% n vertices on one side
% n-1 on the other


% smallest possible bad example
% treecc cc cc cc
% 9 / 13

\subsection*{Insights}

\end{document}