\documentclass{article}

\usepackage{amsmath}
\usepackage{amssymb}
\usepackage{amsthm}

\newtheorem*{theorem*}{Theorem}
\newtheorem*{lemma*}{Lemma}

\newcommand{\opt}{\mathrm{OPT}}

\setlength{\parindent}{0pt}

\begin{document}

\section*{Exercise 1.3}

\subsection*{Problem}

Consider the following factor 2 approximation algorithm for the cardinality vertex cover
problem. Find a depth first search tree in the given graph, $G = (V, E)$, and output the 
set, say $S$, of all the non-leaf vertices of this tree. Show that $S$ is indeed a 
vertex cover for $G$ and $|S| \leq 2 |\opt|$. \\

\textbf{Vertex Cover.} A vertex cover of a graph $G = (V, E)$ is a set of vertices $S$
such that each edge $e \in E$ is incident to at least one vertex $v \in S$. \\

\textbf{Cardinality Vertex Cover Problem.} Given a graph $G = (V, E)$, compute the 
minimum cardinality vertex cover.

\subsection*{Solution}

Define $v_1, v_2, ..., v_n \in V$ to be the depth first search order of the vertices. 
That is, if $i < j$, then $v_i$ occurs before $v_j$ in the depth first search tree. \\

\noindent
Set $M \gets \emptyset$ and $i \gets 1$. While $i < n$
\begin{enumerate}
    \item If $v_i$ is not a leaf, set $M \gets M \cup \{(v_i, v_{i+1})\}$ and 
    $i \gets i + 2$.
    \item Otherwise, set $i \gets i + 1$.
\end{enumerate}
repeat the above.

\begin{lemma*}
    In the depth first search tree of $G = (V, E)$, there are no edges between leaf 
    nodes.
\end{lemma*}

\begin{proof}
    Let $L$ be the set of leaf vertices in the depth first search tree. For the sake of
    contradiction, suppose not, suppose there exist some edge $\{u, v\} \in E$ such that
    $u, v \in L$. Either $u$ or $v$ was explored first in the depth first search tree.
    Without loss of generality, assume $u$ was explored first. Then $v$ must be a child
    of $u$ in the depth first search tree since, when $u$ was encounted, $v$ was 
    unexplored and reachable from $u$. Thus $u$ is not a leaf, which is a contradiction. 
    There must be no edges between leaf nodes in the depth first search tree.
\end{proof}

\begin{lemma*}
    In the depth first search tree of $G = (V, E)$, the set of non-leaf vertices $S$ is a 
    vertex cover of $G$.
\end{lemma*}

\begin{proof}
    Fix some edge $(u, v) \in E$. By the lemma, $u$ or $v$ is a non-leaf vertex so 
    $u \in S$ or $v \in S$. Hence $(u, v)$ is covered so $S$ is a vertex cover.
\end{proof}

\begin{theorem*}
    The set of edges $M$ is a maximal matching.
\end{theorem*}

\begin{proof}
    Since each vertex is added at most once and the edge $\{v_i, v_{i+1}\}$ exists in 
    the depth-first tree for non-leaf $v_i$, $M$ is a matching. The only unmatched 
    vertices are leaves, and by the lemma, there are no edges between them. Hence $M$ is
    a maximal matching.
\end{proof}

\begin{theorem*}
    The set of non-leaf vertices $S$ in the depth first search tree of $G = (V, E)$ is a 
    factor 2 approximation for the minimum cardinality vertex cover of $G$.
\end{theorem*}

\begin{proof}
    By the theorem, there is a maximal matching of size at least $\frac{|S|}{2}$ 
    edges. Since each vertex is incident to exactly one edge in $M$, taking one vertex
    from each edge in $M$ yields a vertex cover of size at least $\frac{|S|}{2}$. Hence
    $\frac{|S|}{2} \leq \opt$ so $|S| \leq 2 \opt$ as desired.
\end{proof}

\subsection*{Insights}

There are no edges between the leaves in a depth-first search tree. \\

Once a lower bounding scheme is found, it can be used to prove bounds for new 
algorithms. In this example, maximal matching provided a known bound on cardinality 
vertex cover, allowing the approximation factor to be proven by constructing a maximal 
matching from the algorithm's output. 

\end{document}
