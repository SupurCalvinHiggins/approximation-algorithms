\documentclass{article}

\usepackage{amsmath}
\usepackage{amssymb}
\usepackage{amsthm}

\newtheorem*{theorem*}{Theorem}
\newtheorem*{lemma*}{Lemma}

\newcommand{\opt}{\mathrm{OPT}}

\setlength{\parindent}{0pt}

\begin{document}

\section*{Exercise 1.1}

\subsection*{Problem}

Give a factor $\frac{1}{2}$ algorithm for the acyclic subgraph problem. \\

\noindent
\textbf{Acyclic Subgraph Problem.} Given a directed graph $G = (V, E)$, pick a maximum 
cardinality set of edges from $E$ so that the resulting subgraph is acyclic.

\subsection*{Solution}

Arbitrarily label the vertices $v_1, v_2, \dots, v_n$. Define
\begin{align*}
    F = \{(v_i, v_j) : i < j \textrm{ and } (v_i, v_j) \in E\}
\end{align*}
and
\begin{align*}
    B = \{(v_i, v_j) : i > j \textrm{ and } (v_i, v_j) \in E\}
\end{align*}
so that $F$ is the set of forwards edges and $B$ is the set of backwards edges. Output 
the larger of the two sets, that is, output $F$ if $|F| \geq |B|$ otherwise output $B$. 

\begin{lemma*}
    Let $G[F]$ be the subgraph induced by $F$ and $G[B]$ be the subgraph induced by $B$.
    Then $G[F]$ and $G[B]$ are acyclic subgraphs of $G$.
\end{lemma*}

\begin{proof}
    For the sake of contradiction, suppose $G[F]$ is cyclic. Since $G[F]$ is cyclic, 
    there exists a sequence of vertices 
    \begin{align*}
        v_{i_1}, v_{i_2}, v_{i_3}, \dots, v_{i_m}
    \end{align*}
    such that the edges
    \begin{align*}
        (v_{i_1}, v_{i_2}), (v_{i_2}, v_{i_3}), \dots, (v_{i_{m-1}}, v_{i_m}), (v_{i_m}, v_{i_1})
    \end{align*}
    are in $F$. Therefore, by definition of $F$, 
    \begin{align*}
        i_1 < i_2 < i_3 < \dots < i_m < i_1
    \end{align*}
    which is a contradiction. Hence $G[F]$ is an acyclic subgraph of $G$. The proof for 
    $G[B]$ is similar.
\end{proof}

\begin{theorem*}
    The largest set between $F$ and $B$ is a $\frac{1}{2}$ factor approximation.
\end{theorem*}

\begin{proof}
    Without loss of generality, assume $F$ is larger than $B$. Since $F$ and $B$ are a
    partition of $E$, $|F| \geq \frac{|E|}{2}$. Also $|E| \geq |\opt|$ so 
    $\frac{|E|}{2} \geq \frac{|\opt|}{2}$. Hence 
    \begin{align*}
        |F| \geq \frac{|E|}{2} \geq \frac{|\opt|}{2}
    \end{align*}
    as desired.
\end{proof}

\subsection*{Insights}

To design a factor $\frac{1}{2}$ algorithm, partition the problem into 2 viable
solutions and take the better of the two. One could also try finding any small set of 
viable solutions such that there must be at least one solution of the desired weight. 
For example, if the average of the viable solutions is at least $\frac{\opt}{3}$, then
taking the best viable solution gives a factor $\frac{1}{3}$ algorithm.

\end{document}
