\documentclass{article}

\usepackage{amsmath}
\usepackage{amssymb}
\usepackage{amsthm}

\newtheorem*{theorem*}{Theorem}
\newtheorem*{lemma*}{Lemma}

\newcommand{\opt}{\mathrm{OPT}}

\setlength{\parindent}{0pt}

\begin{document}

\section*{Exercise 1.6}

\subsection*{Problem}

Give a lower bounding scheme for the arbitrary cost version of the vertex cover problem. 
\\

\textbf{Vertex Cover Problem.} Given an undirected graph $G = (V, E)$ and a cost 
function $c : V \rightarrow \mathbb{Q}^+$ find the minimum cost vertex cover, where the
cost of a cover $U$ is $c(U) = \sum\limits_{u \in U} c(u)$.

\subsection*{Solution}

The vertex cover problem is equivalent to the following 
\begin{align*}
    \text{minimize} \quad & \sum\limits_{v \in V} x_v c(v) \\
    \text{subject to} \quad
        & x_u + x_v \geq 1 \quad & \forall (u, v) \in E \\
        & x_v \in \{0, 1\} \quad & \forall v \in V
\end{align*}
integer linear program. The variable $x_v$ takes the value $1$ if and only if $v$ is in 
the computed vertex cover. \\

The ILP has the following
\begin{align*}
    \text{minimize} \quad & \sum\limits_{v \in V} x_v c(v) \\
    \text{subject to} \quad
        & x_u + x_v \geq 1 \quad & \forall (u, v) \in E \\
        & x_v \geq 0 \quad & \forall v \in V
\end{align*}
linear programming relaxation. Define
\begin{align*}
    U = \{v : x_v \geq \frac{1}{2} \text{ and } v \in V\}
\end{align*}
to be the solution to vertex cover based on the LP-relaxation.

\begin{theorem*}
    $U$ is a vertex cover.
\end{theorem*}

\begin{proof}
    Fix some arbitrary edge $(u, v) \in E$. Since $x_u + x_v \geq 1$ and 
    $x_u, x_v \geq 0$, $x_u \geq \frac{1}{2}$ or $x_v \geq \frac{1}{2}$. Then $u \in V$
    or $v \in V$ so $(u, v)$ is covered. Hence $U$ is a vertex cover.
\end{proof}

\begin{theorem*}
    $U$ is a factor 2 approximation for the LP-relaxation.
\end{theorem*}

\begin{proof}
    We have that
    \begin{align*}
        c(U) = \sum\limits_{u \in U} c(u) \leq 2 \sum\limits_{u \in U} c(u) x_u \leq 
        2 \sum\limits_{v \in V} c(v) x_v = 2 \opt_{LP}
    \end{align*}
    as desired.
\end{proof}

\begin{theorem*}
    $U$ is a factor 2 approximation for vertex cover.
\end{theorem*}

\begin{proof}
    We have that
    \begin{align*}
        c(U) \leq 2 \opt_{LP} \leq 2 \opt
    \end{align*}
    as desired.
\end{proof}

\subsection*{Insights}

To obtain a lower bounding scheme for a problem, formulate the problem as an integer 
linear program and compute the linear programming relaxation. From the solution to the 
LP-relaxation, re-construct a solution to the original problem via rounding. Consider 
the approximation factor, say $\alpha$, the rounded solution obtains with respect to the 
LP-relaxation. Since $\alpha \opt_{LP} \leq \alpha \opt$, analyzing an algorithm with 
respect to the LP-relaxation can prove approximation factor no better than $\alpha$. \\

\noindent
Taking the rounded LP-relaxation as the approximation algorithm yields a polynomial time
approximation algorithm with factor $\alpha$. However, non-LP algorithms might be faster
in practice (e.g. lower, but still polynomial, time complexity).

\end{document}